% !TeX program = lualatex

\documentclass[
    tikz,
    margin=0.5cm,
    convert,
]{standalone}

% Arrow heads
\usetikzlibrary{arrows.meta}
\usetikzlibrary {graphs}
\usetikzlibrary{quotes}
\usetikzlibrary{positioning}
\usetikzlibrary{graphdrawing}
\usegdlibrary{layered}

% Font
\usepackage{roboto-mono}
\usepackage{roboto}

\newcommand{\class}[1]{\texttt{\textbf{#1}}}

\begin{document}
    \begin{tikzpicture}[
        font=\sffamily
        ]
        \graph [
            layered layout,
            level distance=5cm,
            sibling distance=5cm,
            simple,
            edge quotes={
                anchor=center,
                fill=black!15,
                outer sep=-1.5ex,
                rounded corners=0.25em,
                left/.append style={align=right},
                right/.append style={align=left},
                align=center,
                inner sep=0.5em,
            },
            nodes={
                rectangle,
                draw=black,
                fill=black!20,
                inner sep=1em,
                align=center,
                rounded corners=0.25em,
            }
        ] {
            Bot[
                as={
                    \class{(ext.Ext)Bot}\\
                    Bot API client\\
                    used to make requests\\
                    to the API
                }
            ] -> {
                BaseRateLimiter[
                    as={
                        \class{ext.BaseRateLimiter}\\
                        Interface for rate limiting\\
                        API requests
                    },
                    >"dispatches requests\\to the API",
                ] -> {
                    BaseRequest[
                        as={
                            \class{request.BaseRequest}\\
                            Inferface for handling the\\
                            networking backend
                        },
                        >"rate limits requests\\to the API",
                    ]
                },
                Defaults[
                    as={
                        \class{ext.Defaults}\\
                        Gathers default values for frequently\\
                        used parameters of \class{(ext.Ext)Bot},\\ \class{ext.JobQueue} and \class{ext.Handler}
                    },
                    >"gets default values\\for parameters",
                ],
                CallbackDataCache[
                    as={
                        \class{ext.CallbackDataCache}\\
                        In-Memory LRU-cache for\\
                        arbitrary \texttt{callback\_data}
                    },
                ]
            };
            Bot ->[bend right] BaseRateLimiter;
            Bot[
                <"stores arbitrary\\\texttt{callback\_data}"
            ] ->[bend left] CallbackDataCache;
            Application[
                text width=60ex,
                as={
                    \class{ext.Application}
                    \begin{itemize}
                        \item entry point for the whole application
                        \item provides convenience methods for running the whole app via \texttt{run\_polling/webhook()}
                        \item administers handlers and error handlers
                        \item administers \texttt{user/chat/bot\_data}
                        \item Very similar to the previous \texttt{Dispatcher}
                    \end{itemize}
                },
                level post sep=3cm,
            ] -> {
                Updater[
                    as={
                        \class{ext.Updater}\\
                        fetches updates from Telegram\\
                        and puts them into the \texttt{update\_queue}
                    },
                    >"fetches updates from\\the \texttt{update\_queue}"
                ],
                Handler[
                    as={
                        \class{ext.Handler}\\
                        Specifies if and how\\it handles updates
                    },
                    >"provides arguments for\\handler callbacks,\\processes exceptions\\raised in handler callbacks",
                ],
                BasePersistence[
                    as={
                        \class{ext.BasePersistence}\\
                        Interface for persisting\\
                        data from \class{ext.Application}\\
                        across restarts
                    },
                    >"updates in\\regular intervals",
                ],
                ContextTypes[
                    as={
                        \class{ext.ContextTypes}\\
                        Specifies types of\\
                        the \texttt{context} argument
                    },
                    >"accesses to\\build \texttt{context}",
                ],
                JobQueue[
                    as={
                        \class{ext.JobQueue}\\
                        Schedules tasks to run\\
                        at specified times
                    },
                    >"processes exceptions\\raised in jobs",
                ], 
                CallbackDataCache[
                    >"Fetches data and\\passes it to\\\class{ext.BasePersistence}"
                ]
            };
            Application ->[pos=0.3] Updater;
            Application ->[near end] BasePersistence;
            Application ->[pos=-0.25] CallbackDataCache;
            Application ->[bend left, near end] JobQueue;
            Application ->[bend left=60] ContextTypes;
            Application ->[pos=0.55, bend right] Defaults;
            ApplicationBuilder[
                as={
                    \class{ext.ApplicationBuilder}\\
                    builder pattern for\\
                    \class{ext.Application}
                },
                <"builds",
            ] -> Application;
            Updater[
                <"calls \texttt{get\_updates}\\
                \& \texttt{set/delete\_webhook}"
            ] -> Bot;
            BasePersistence[
                <"holds a\\reference"
            ] -> Bot;
            JobQueue -> ContextTypes;
            ContextTypes -> CallbackContext[
                as={
                    \class{ext.CallbackContext}\\
                    Convenience class for unified\\
                    access to different objects within\\
                    handler/job/error callbacks
                },
                >"specifies types"
            ]
        };


        % Title
        \node[above=2.5em of current bounding box.north,font=\huge\bfseries\sffamily] {Overview of the architecture of \texttt{python-telegram-bot}};
    \end{tikzpicture}
\end{document}